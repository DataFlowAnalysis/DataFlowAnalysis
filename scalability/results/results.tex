\documentclass{article}
\usepackage{pgfplots}
\pgfplotsset{compat=newest}

\pgfplotsset{
    box plot/.style={
        /pgfplots/.cd,
        black,
        only marks,
        mark=-,
        mark size=\pgfkeysvalueof{/pgfplots/box plot width},
        /pgfplots/error bars/y dir=plus,
        /pgfplots/error bars/y explicit,
        /pgfplots/table/x index=\pgfkeysvalueof{/pgfplots/box plot x index},
    },
    box plot box/.style={
        /pgfplots/error bars/draw error bar/.code 2 args={%
            \draw  ##1 -- ++(\pgfkeysvalueof{/pgfplots/box plot width},0pt) |- ##2 -- ++(-\pgfkeysvalueof{/pgfplots/box plot width},0pt) |- ##1 -- cycle;
        },
        /pgfplots/table/.cd,
        y index=\pgfkeysvalueof{/pgfplots/box plot box top index},
        y error expr={
            \thisrowno{\pgfkeysvalueof{/pgfplots/box plot box bottom index}}
            - \thisrowno{\pgfkeysvalueof{/pgfplots/box plot box top index}}
        },
        /pgfplots/box plot
    },
    box plot top whisker/.style={
        /pgfplots/error bars/draw error bar/.code 2 args={%
            \pgfkeysgetvalue{/pgfplots/error bars/error mark}%
            {\pgfplotserrorbarsmark}%
            \pgfkeysgetvalue{/pgfplots/error bars/error mark options}%
            {\pgfplotserrorbarsmarkopts}%
            \path ##1 -- ##2;
        },
        /pgfplots/table/.cd,
        y index=\pgfkeysvalueof{/pgfplots/box plot whisker top index},
        y error expr={
            \thisrowno{\pgfkeysvalueof{/pgfplots/box plot box top index}}
            - \thisrowno{\pgfkeysvalueof{/pgfplots/box plot whisker top index}}
        },
        /pgfplots/box plot
    },
    box plot bottom whisker/.style={
        /pgfplots/error bars/draw error bar/.code 2 args={%
            \pgfkeysgetvalue{/pgfplots/error bars/error mark}%
            {\pgfplotserrorbarsmark}%
            \pgfkeysgetvalue{/pgfplots/error bars/error mark options}%
            {\pgfplotserrorbarsmarkopts}%
            \path ##1 -- ##2;
        },
        /pgfplots/table/.cd,
        y index=\pgfkeysvalueof{/pgfplots/box plot whisker bottom index},
        y error expr={
            \thisrowno{\pgfkeysvalueof{/pgfplots/box plot box bottom index}}
            - \thisrowno{\pgfkeysvalueof{/pgfplots/box plot whisker bottom index}}
        },
        /pgfplots/box plot
    },
    box plot median/.style={
        /pgfplots/box plot,
        /pgfplots/table/y index=\pgfkeysvalueof{/pgfplots/box plot median index}
    },
    box plot width/.initial=1em,
    box plot x index/.initial=0,
    box plot median index/.initial=1,
    box plot box top index/.initial=2,
    box plot box bottom index/.initial=3,
    box plot whisker top index/.initial=4,
    box plot whisker bottom index/.initial=5,
}

\newcommand{\boxplot}[2][]{
    \addplot [box plot median,#1] table [col sep=comma]{#2};
    \addplot [forget plot, box plot box,#1] table [col sep=comma]{#2};
    \addplot [forget plot, box plot top whisker,#1] table [col sep=comma]{#2};
    \addplot [forget plot, box plot bottom whisker,#1] table [col sep=comma]{#2};
}

\begin{document}
\begin{figure}
	\begin{tikzpicture}
		\begin{axis}[xmode=log, ymode=log, log ticks with fixed point]
			\boxplot [
				forget plot,
				box plot whisker bottom index=5,
				box plot whisker top index=4,
				box plot box bottom index=3,
				box plot box top index=2,
				box plot median index=1
			]{NewCharacteristicsPropagationTest.csv}
		\end{axis}
	\end{tikzpicture}
	\caption{New Analysis - CharacteristicsPropagation}
\end{figure}

\begin{figure}
	\begin{tikzpicture}
		\begin{axis}[xmode=log, ymode=log, log ticks with fixed point, xlabel=Index, ylabel=Time, legend pos= north west]
			\addplot table [x=index, y=median, col sep=comma] {NewCharacteristicsPropagationTest.csv};
			\addlegendentry{New Analysis}
			\addplot table [x=index, y=median, col sep=comma] {OldCharacteristicsPropagationTest.csv};
			\addlegendentry{Old Analysis}
			\addplot[mark=none, black, dotted, thick, samples=2, domain=1:1000000]{x};
		\end{axis}
	\end{tikzpicture}
	\caption{CharacteristicsPropagation}
\end{figure}

\begin{figure}
	\begin{tikzpicture}
		\begin{axis}[xmode=log, ymode=log, log ticks with fixed point, xlabel=Index, ylabel=Time, legend pos= north west]
			\addplot table [x=index, y=median, col sep=comma] {NewNodeCharacteristicsTest.csv};
			\addlegendentry{New Analysis}
			\addplot table [x=index, y=median, col sep=comma] {OldNodeCharacteristicsTest.csv};
			\addlegendentry{Old Analysis}
			\addplot[mark=none, black, dotted, thick, samples=2, domain=1:1000000]{x};
		\end{axis}
	\end{tikzpicture}
	\caption{NodeCharacteristics}
\end{figure}

\begin{figure}
	\begin{tikzpicture}
		\begin{axis}[xmode=log, ymode=log, log ticks with fixed point, xlabel=Index, ylabel=Time, legend pos= north west]
			\addplot table [x=index, y=median, col sep=comma] {NewSEFFParameterTest.csv};
			\addlegendentry{New Analysis}
			\addplot table [x=index, y=median, col sep=comma] {OldSEFFParameterTest.csv};
			\addlegendentry{Old Analysis}
			\addplot[mark=none, black, dotted, thick, samples=2, domain=1:1000000]{x};
		\end{axis}
	\end{tikzpicture}
	\caption{SEFFParameter}
\end{figure}

\begin{figure}
	\begin{tikzpicture}
		\begin{axis}[xmode=log, ymode=log, log ticks with fixed point, xlabel=Index, ylabel=Time, legend pos= north west]
			\addplot table [x=index, y=median, col sep=comma] {NewVariableActionsTest.csv};
			\addlegendentry{New Analysis}
			\addplot table [x=index, y=median, col sep=comma] {OldVariableActionsTest.csv};
			\addlegendentry{Old Analysis}
			\addplot[mark=none, black, dotted, thick, samples=2, domain=1:1000000]{x};
		\end{axis}
	\end{tikzpicture}
	\caption{VariableActions}
\end{figure}


\begin{figure}
	\begin{tikzpicture}
		\begin{axis}[xmode=log, ymode=log, log ticks with fixed point, xlabel=Index, ylabel=Time, legend pos= north west]
			\addplot table [x=index, y=median, col sep=comma] {NewVariableCountTest.csv};
			\addlegendentry{New Analysis}
			\addplot table [x=index, y=median, col sep=comma] {OldVariableCountTest.csv};
			\addlegendentry{Old Analysis}
			\addplot[mark=none, black, dotted, thick, samples=2, domain=1:1000000]{x};
		\end{axis}
	\end{tikzpicture}
	\caption{VariableCount}
\end{figure}

\end{document}














































